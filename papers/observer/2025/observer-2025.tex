\documentclass[twocolumn]{article}
\usepackage[utf8]{inputenc}
\usepackage{graphicx}
\usepackage{amsmath}
\usepackage{amssymb}
\usepackage{hyperref}
\usepackage{lipsum}

% Title and author information
\title{An Example of a Long Paper}
\author{John Doe \\
Department of Computer Science, Example University \\
\texttt{johndoe@example.com}}
\date{January 2025}

\begin{document}

    \maketitle

    \begin{abstract}
        This paper demonstrates the structure of a typical scientific article written in \LaTeX. It includes an abstract, multiple sections, a figure, a table, and references. The content is generated for educational purposes and is intended to showcase a well-formatted, multi-page document.
    \end{abstract}

    \section{Introduction}
    Scientific writing requires clarity, precision, and a well-organized structure. This document demonstrates how to use \LaTeX\ to create professional papers. The introduction typically sets the stage for the research, outlines the problem, and highlights the paper's contributions.

    \lipsum[1-2]

    \section{Background}
    In this section, relevant background information is provided. This includes prior research, definitions, and the context necessary to understand the rest of the paper.

    \lipsum[3-4]

    \section{Methodology}
    The methodology describes the approach, data collection, and analysis techniques used in the research. Here, equations and figures can be used to clarify the process.

    \subsection{Mathematical Model}
    The following equation illustrates a fundamental concept:
    \begin{equation}
        E = mc^2,
    \end{equation}
    where $E$ represents energy, $m$ is mass, and $c$ is the speed of light.

    \subsection{Data Collection}
    Data collection was performed using standard techniques. The process is summarized in Table~\ref{tab:data}.

    \begin{table}[h]
        \centering
        \begin{tabular}{|c|c|c|}
            \hline
            Sample & Value & Category \\
            \hline
            1      & 23.4  & A        \\
            2      & 19.8  & B        \\
            3      & 45.2  & C        \\
            \hline
        \end{tabular}
        \caption{Summary of data collection.}
        \label{tab:data}
    \end{table}

    \section{Results and Discussion}
    The results are presented and discussed in this section. Visual aids like graphs and images enhance understanding.


    \lipsum[5-6]

    \section{Conclusion}
    This paper demonstrates how to format a scientific article in \LaTeX. By including sections, figures, tables, and references, authors can create professional and impactful documents.

    \lipsum[7-8]

    \section*{References}
    \begin{thebibliography}{9}
        \bibitem{example2025} A. Author, B. Researcher, ``A Comprehensive Guide to \LaTeX,'' Example Press, 2025.

        \bibitem{knuth1984} D. E. Knuth, ``The book,'' Addison-Wesley, 1984.

        \bibitem{lamport1994} L. Lamport, ``\LaTeX: A Document Preparation System,'' Addison-Wesley, 1994.
    \end{thebibliography}

\end{document}
